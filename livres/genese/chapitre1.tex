\bchapter[chapitre]

\begin{chaptercomment}
 \footnotemarkchapter{}Creation du ciel \& de la terre,
 \emph{II, 20.} \& de tout ce qui y est com- \lb
 prins.  \emph{3.14} De la lumiere aussi,
 \emph{26} \& de l'homme,
 \emph{28} {\addfontfeature{RawFeature=+swsh}Auquel}\lb
 tout est assubietti.
 \emph{2.2.} \emph{28} Dieu benit toutes ses \oe{}uvres,
 \emph{31} qu'il \lb
 a accomplies en six iours.
\end{chaptercomment}

\vspace{\baselineskip}

% TBD: span lettrine on 10 lines
\bversenonum \lettrine[lines=10,image=true,loversize=0.05,lraise=-0.03]{D}{}%
 \footnotemarkverse{}Ieu
 \footnotemarkmain{}crea
 \footnotemarkmain{}au com \lb
 mence - ment
 \footnotemarkmain{}le ciel \& la terre.
\bversenopar[]Or la\lb
 terre eſ-\lb toit sans forme, \& \lb
 vuide,\& les tenebres estoyent sur les
abysmes~: \& l'Esprit de Dieu
 \footnotemarkmain{}estoit
 espandu par dessus les eaux.



\bverse Adonc Dieu dît,
 \footnotemarkverse{}Qu'il y ait~lumie\-re.
 \footnotemarkmain{}Et la lumiere fut.

\bverse Et Dieu vid \~q la lumiere estoit bon\-ne~: \& separa la lumiere des tenebres.

\bverse Et Dieu appela la lumiere iour,\& les
 tenebres nuict. Lors fut faict le
 \footnotemarkmain{}soir \& le matin 
 du premier iour.

\bverse ¶ Puis Dieu dît,
 \footnotemarkverse{}Qu'il y ait une
 \footnotemarkmain{}eſ-\lb
 tendue entre les eaux,
 \& qu'elle separe les
 \footnotemarkmain{}eaux d'avec les eaux.

\bverse Dieu donc fit l'estendue, \& divisa\lb

% Notes for the page
\marginnotes{-6.0in}{
   \fakenote{I}{Ce premier cha- \lb pitre est fort diffi- \lb
 cile~: \& pour cette \lb
 cause, il estoit de- \lb
 fendu entre les He \lb
 brieux de le lire \& \lb
 interpreter devant \lb
 l'aage de trente \lb ans.}
   \fakenote{a}{Fit de rien, \& \lb sans aucune ma- tiere.}
   \fakenote{1}{\bibleverse{Job}(38:4), \bibleverse{Ps} \lb
   33.6, \textit\& 89.12.,\lb 135.5, \bibleverse{Ec} \lb
   13.1, \bibleverse{Ac} 14-15, \lb
   \textit\& 17.14}
   \fakenote{b}{Tout premierement, \& av\~at qu'il y eut aucune
 creature, \bibleverse{Jn}(1:10).}
   \fakenote{2}{\bibleverse{He}(11:3).}
   \fakenote{c}{Le ciel \& la terre,
 les eaux, les \lb
 abysmes, se pren- nent ici pour vne mesme chose~: aſç. \lb
 pour une matiere c\~ofuse \& sans for- \lb
 me, \~q Dieu forma \lb
 \& agença apres \lb
 par sa Parole.}
   \fakenote{d}{Ou, se mou- \lb
 voit. C'est, souste- \lb
 noit et conservoit \lb
 en son estre cette \lb
 matiere confuse. \\
 Car il est impossi- \lb
 ble, \~q aucune cho- \lb
 se apres avoir esté \lb
 faictes,puisse subsi- \lb
 ster un seul mo- \lb
 ment, si Dieu ne la \lb
 soustient \& c\~oser- \lb
 ve par sa vertu, \lb
 \bibleverse{Ps}(130:).}
   \fakenote{e}{Cette lumiere \lb
 n'estoit point en- \lb
 core au soleil, car \lb
 il n'avoit pas esté \lb
 creé, mais estoit en \lb
 la main de Dieu, \lb
 ay\~at son ordre suc- \lb
 cessif avec les tene- \lb
 bres, pour faire le \lb
 iour \& la nuict \& \lb
 ce iusques au qua- \lb
 trieme iour, que \lb
 Dieu fit le soleil \lb
 pour estre ministre \lb
 \& dispensateur de \lb
 cette lumiere, avec \lb
 la lune \& estoilles.}
   \fakenote{3}{\bibleverse{Ps}(33:6), \textit\& 136.5.\\
                \bibleverse{Je}(10:11) \textit\& 51.15.}
   \fakenote{f}{Ici est la cause}
}
 \pagebreak

 \noindent les eaux, qui estoyent sous
 l'estendue, d'avec celles, qui estoyent sur l'esten\-due. Et fut ainsi faict.

\bverse Et Dieu appela l'estendue, Ciel.
 Lors fut faict le soir \& le matin du second iour.

\bverse ¶ Puis Dieu dît, 
 \footnotemarkverse{}~\footnotemarkmain{}Que les eaux,
 qui sont sous le ciel,
 soyent assemblees en un lieu, \& que le sec apparoisse. Et fut ainsi faict.

\bverse Et Dieu appe\char"A749 ale sec,Terre,\& l'assem \lb
 blee des eaux, mers.
 Et Dieu vid que celà estoit bon.

\bverse Et Dieu dît, Que la terre produise verdure, herbe produisant semence,
 \& arbre fruictier, faisant fruict selon son espece, lequel ait sa sem\~ece
 en soy-meſ\-me sur la terre. Et fut ainsi faict.

\bverse La terre d\~oc produisit verdure, her\-be produisant sem\~ece
 selon son espece, \& arbre sans fruict, lequel avoit sa \lb
  semence en soymesme selon son espe- \lb
 ce. Et Dieu vid que celà estoit bon.

\bverse Lors fut faict le soir \& le matin du troisieme iour.

%skip j counter
% TBD: do it better
\addtocounter{footnotemain}{1}

\bverse ¶ Apres Dieu dît,\footnotemarkverse{}\,\footnotemarkmain{}Qu'il y ait lumi \lb
 naires en l'estendue du ciel, pour
 sepa\-rer la nuict du iour~: \& soy\~et en
 \footnotemarkmain{}signes,

\vspace{-2cm}
\begin{flushright}
a\quad en
\end{flushright}

% Notes for the page
\marginnotes{-16.3cm}{
 % Continue note f from left column
 pourquoy les Hebrieux c\~omencent
 le iour naturel le soir apres le soleil couchant.
   \fakenote{g}{Ce mot d'Est\~e \lb
 due, compr\~ed tout ce qui
 se voit par dessus nous, t\~at en la region celeste,
 qu'elementaire.}

   \fakenote{4}{\bibleverse{Ps}(33:7).}
   \fakenote{h}{Il est ici parlé de deux manieres
 d'eaux~: asçavoir, celles q sont sous
 l'estendue, comme la mer, les fleuves,
 \& autres qui sont sur la terre \& cel- \lb
 les, qui sont sur l'estendue,
 comme sont les nuees plei- \lb
 nes d'eau
 ça haut en l'air par dessus nous.
 Dieu a mis \lb
 entre ces deux for \lb
 ces d'eaux une gr\~a \lb
 de estendue, qu'on \lb
 appelle le ciel~: de \lb
 là nous appelons les oiseaux du ciel.}
   \fakenote{i}{Ceci apparti\~et au sec\~od iour,
 auquel Dieu separa, \& fit apparoir la terre du milieu des eaux.}
   \fakenote{k}{Il institue un nouvel ordre
 en nature, quand il faut \& ordonne le soleil distributeur
 de cette lumiere qu'il avoit creée avant lui, \& avant la lune \& les eſ- \lb
 toilles.}
   \fakenote{5}{\bibleverse{Ps}(136:7)}
   \fakenote{l}{C'est pour si- \lb
 gnifier diverses di- \lb
 spositions que les corps \~iferieurs se- \lb
 lon l'ordre de na- \lb 
 ture ont des corps \lb
 celestes, c\~ome cau \lb
 ses sec\~odes ordon \lb
 nees de Dieu à ce- \lb 
 là.  En quoy tou- \lb
 tesfois faut fuir cu- \lb
 riosité \& supersti- \lb
 tion \~q les h\~omes \lb
 ont c\~otrouvee sur \lb
 celà.}
}

% Page 2
\pagebreak

\footnotemarkmain{} en saisons,
\footnotemarkmain{}en iours,\& en ans.

\bverse Et soyent pour luminaires au firma\-ment
du ciel, à fin de donner lumiere
sur la terre. Et fut faict ainsi.

\bverse Dieu donc fit deux grans luminai\-res:le
 plus grand luminaire pour
\footnotemarkmain{} gouverner le iour,\& le moindre
pour gou \lb
verner la nuict:\& les estoilles.

\bverse Et Dieu les mit en l'estendue du ciel,
pour luire sur la terre,

\bverse Et pour gouverner le iour, \& la
nuict,\& pour separer la lumiere des te \lb
nebres.Et Dieu vid que celà estoit b\~o.

\bverse Lors fut faict le soir \& le matin du
quatrieme iour.

\bverse ¶ En apres Dieu dît, Que les~eaux
produisent abond\~ament
\footnotemarkmain{}reptile ayant ame
vivante : \& volaille vole sur la ter\-re
envers l'estendue du ciel.

\bverse Dieu donc
\footnotemarkmain{}crea de gr\~ades baleines,
\& toute creature vivante se mouvant,
que les eaux produirent selon leur es\-pece :
\& toute volaille ayant ailes chacune selon son espece : 
\& Dieu vid \lb
 que celà estoit bon.

\bverse Adonc \footnotemarkmain{}il les benit, disant, Fructi\-fiez,
\& multipliez, \& remplissez les
eaux des mers:\& que la volaille se mul \lb
tiplie en la terre.

\bverse Lors fut faict le soir \& le matin du
cinquieme iour.

\bverse ¶ Outre Dieu dît,Que la terre produise
creature vivante selon son espe\-ce,~bestail
 \& reptile, \& animau de la
terre chacun selon leur espece. Et fut
ainsi faict.

\bverse Dieu donc fit l'animau de la terre
selon son espece, \& le bestail selon son
espece, \& tout le reptile de la terre se\-lon
son espece. Et Dieu vid que celà
estoit bon.

\bverse Outreplus Dieu dît,
\footnotemarkmain{}Faisons
\footnotemarkmain{}l'h\~o\-me
\addtocounter{footnotemain}{1}% Pas de note u
à \footnotemarkverse{}\footnotemarkmain{} nostre image,
\& selon nostre
semblance, \& qu'il ait domination sur
les poissons de la mer, \& sur les oi\-seaux
du ciel, \& sur les bestes, \& sur
toute la terre, \& sur tout reptile ram\-pant
sur la terre.
\addtocounter{footnotemain}{1}% Pas de note w

\bverse Dieu donc crea l'homme à son ima \lb
ge:\footnotemarkverse{}\footnotemarkmain{}il les
crea,\emph{di-ie},à l'image de Dieu:
\footnotemarkverse{}il les crea masle \& femelle.

\bverse Et Dieu les benit,\& leur dît,\footnotemarkmain{}Fructi

% Notes de gauche
\marginnotes{-23.5cm}{
  \fakenote{m}{Ce mot deno-\lb
  te la distincti\~o \& \lb
  diversité des t\~eps \lb
  ord\~onés pour l'u- \lb
  sage des hommes: \lb
  soit pour le re- \lb
  gard de la religi\~o, \lb
  comme on voit à l'ordonnance des \lb
  solennités de Paſ- \lb
  ques, Pentecoste, \lb
  Tabernacles, Nou \lb
  velles lunes , ou \lb
  pour le regard de \lb
  la police, tant au public qu'au par- \lb
  ticulier.}
  \fakenote{n}{Qui sont dep\~e \lb
  dans du mouve- \lb
  ment du soleil. Or \lb
  le S.Esprit ne de- \lb
  claire point les se\-crets
  de nature \~q \lb
  l'astrologie ensei- \lb
  gne: ains seulem\~et \lb
  propose ce qui est \lb
  devant nos yeux, \lb
  c\~ome doctrine ge \lb
  nerale pour tous. \lb
  Car celui est le \lb
  vray livre,où mes- \lb
  mesles plus idiots \lb
  peuvent appr\~edre \lb
  à cognoistre Dieu.}
  \fakenote{o}{C\~ome instru- \lb
  mens ordonnés à \lb
  celà pour servir à \lb
  l'homme.}
  \fakenote{p}{C'est, beste qui \lb
  se traine ou r\~ape.}
  \fakenote{q}{La matiere de \lb
  ces poissons fut \lb
  creée dés le com- \lb
  mencement, mais \lb
  ils furent formés à \lb
  ce cinquieme iour: \lb
  c'est la cause dont \lb
  il dit, Crea.}
  \fakenote{r}{C'est, leur d\~o- \lb
  na vertu d'engen- \lb
  drer. Ce qui est m\~o \lb
  stré apres où il dit, \lb
  Fructifiez, \&c. les- \lb
  quelles paroles em \lb
  portent c\~omande- \lb
  ment que Dieu par \lb
  sa parole a impri- \lb
  mé à ses creatures, \lb
  dont elles ont ver \lb
  tu perman\~ete ius- \lb
  ques en la fin du \lb
  monde.}
  \fakenote{s}{Il a dict ci des- \lb
  sus, Que les eaux \lb
  produisent. Que la \lb
  terre produise,\&c. \lb
  Maintenant il dit, \lb
  Faisons. En quoy \lb
  nous faut ent\~edre \lb
  une nouvelle ma- \lb
  niere de proceder. \lb
  Comme si Dieu
  entrant en conseil
  \& deliberation, a- \lb
  vec sa sagesse \&
  vertu, proposoit de
  faire un chef d'\oe{}u \lb
  vre excellent par
  dessus les autres creatures.}
  \fakenote{t}{Ce mot d'h\~o- \lb
  me denote qu'il
  ne parle de l'ame
  seulement : car
  l'h\~ome entier en
  sa perfection re- \lb
  presentoit l'excel- \lb
  lence de Dieu.}
  \fakenote{6}{\emph{Sous} 5.1. \emph{\&} 9.6}
  \fakenote{7}{\bibleverse{Ws}(2:23). \bibleverse{Qo}(17:1).
  % \bibleverse{ICo}(11:7). FIXME
  \emph{1.Cor.} 11.7
  \bibleverse{Col}(3:10).}
  \fakenote{8}{
  % FIXME encore une autre abbr
  %\bibleverse{Qo}(17:5).
  \emph{Ecclesiast.} 17.5.
  \bibleverse{Mt}(19:4) \\
  \emph{Sous} 8.17 \emph{et} 9.1}
  \fakenote{u}{Cette image \&
  sembl\~ace de Dieu
  en l'h\~ome est m\~o
  stree par S. Paul,
  \bibleverse{Ep}(4:). quand il
  dit que l'h\~ome est
  creé en vraye sain
  cteté \& iustice,c\~o- \lb
  prenant par ces
  deux mots toute
  la perfection de la
  nature d'icelui:c'est
  aſçavoir, une lu- \lb
  miere de droite in}
}

% Page 2: deuxieme colonne
\vfill\break
\noindent fiez,\& multipliez,\& remplissez la terre,
\& l'assubiettissez : \& \footnotemarkmain{}ayez seigneurie
sur les poissons de la mer, \& sur les oi\-seaux
du ciel, \& sur tous animaux qui se mouvent sur la terre.

\bverse D'avantage Dieu dît, Voici, ie vous
ay donné toute herbe qui produit se\-mence
qui est sur toute la terre,\& tout
arbre qui a en soy fruict d'arbre pro\-
duisant semence : \footnotemarkverse{}à fin qu'ils vous
soyent pour viande.

\bverse Mesmes aussi à tous animaux de la
terre,\& à tous oiseaux du ciel,\& à tou\-te
chose mouv\~ate sur la terre,qui a en
soy ame vivante, toute herbe verde se\-ra
pour viande. Et fut ainsi faict.

\bverse \footnotemarkmain{} \footnotemarkverse{}Et Dieu vid
tout ce qu'il avoit faict, \& voilà il estoit moult bon. Lors
fut faict le soir \& le matin du sixieme
iour.

