\documentclass[twocolumn,paper=8in:13in,pagesize=pdftex,12pt]{scrbook}
\areaset[0.5in]{5.2in}{10.8in}

\usepackage{bigfoot}
\usepackage{ragged2e}

\usepackage{fontspec}
\usepackage{xunicode}
\defaultfontfeatures{Mapping=tex-text}
%\setmainfont{Linux Libertine O}
\setmainfont[RawFeature={+ss02,+cv01,+dlig}]{EB Garamond}
\newfontfamily\booktitlefont[LetterSpace=40,WordSpace=7]{EB Garamond}

\usepackage{graphicx}
\usepackage{lettrine}


% Verse references
% To be adjusted
\usepackage{bibleref-french}
\renewcommand*{\BRchvsep}{.}%
\renewcommand*{\BRvsep}{,}%
\setbooktitle{Job}{Iob}%
\setbooktitle{Ps}{Pseam.}%
\setbooktitle{He}{Hebr.}%
\renewcommand{\BRbooktitlestyle}{\textit}


% Environments
\newenvironment{comment}
  {\itshape\scriptsize\setlength\hspace{-1em}}
  {}


% Footnotes
\DeclareNewFootnote{default}[alph]
\DeclareNewFootnote{chapter}[Roman]
\DeclareNewFootnote{verse}[arabic]


\setlength{\marginparwidth}{4.3em}% adjust to your document's needs

\newcommand{\chapnote}[1]{%
   \footnotemarkchapter%
   \marginpar{\tiny\textsuperscript{\thefootnotechapter}\justifying#1}}

\newcommand{\sidenote}[1]{%
   \footnotemark%
   \marginpar{\tiny\textsuperscript{\thefootnote}\justifying#1}}

\newcommand{\versenote}[1]{%
   \footnotemarkverse%
   \marginpar{\tiny\textsuperscript{\thefootnoteverse}\justifying#1.}}

% Bible verses
\newcounter{verse}
\newcommand{\bverse}{%
  \addtocounter{verse}{1}
  \par\theverse~
}

\newcommand{\bversenopar}{%
   \addtocounter{verse}{1}\\\indent\theverse~
}

\newcommand{\bversenonum}{%
   \addtocounter{verse}{1}
   \par
}


\usepackage{titlesec}
% Use fourier ornaments
\usepackage{fourier-orns}

% Bible books
\newcommand{\bbook}[3][]{%
  \chapter[#1]{#2,\\\Large #3\\\aldine}
}
\titleformat{\chapter}[hang]%
   {\centering\huge}%
   {}%
   {5pt}%
   {}

% Bible chapters
\newcommand{\bchapter}{%
   \setcounter{verse}{0}%
   \section{}{}
}

\renewcommand{\thesection}{\Roman{section}}
\titleformat{\section}[hang]%
   {\booktitlefont\centering}%
   {CHAPITRE\ \thesection}%
   {5pt}%
   {}


\begin{document}

\twocolumn[
\begin{@twocolumnfalse}
\bbook{Le premier livre de Moyse}{Dict Genese.}

\begin{center}
\large ARGVMENT.
\end{center}
\begin{comment}
Ce premier livre comprendre l'origine \& causes de toutes choses, principalement
 la creation de l'homme, qu'il a esté du commencement, sa cheute \& relevement:
 comment d'un tous ont esté procreés, \& pour leurs enormes pechés Dieu
 les a consumés, par le deluge, reservé huict, dont la semence a rempli toute
 la terre. Puis il deserit les vies, faicts, religion, \& lignees des saints
 Patriarches, qui ont veseu devant la Loy: Les benedictions, promesses,
 \& alliances du Seigneur faictes avec iceux : Comment de le la terre de Chanaan
 sont descendus en Egypte. Aucuns ont appelé ce livre, le livre des Iustes.
 Toutefois ceci a obtenu entre nos predecesseurs \& nous, qu'il est appelé
 Genese, qui est un mot Grec, signifiant generation \& origine : d'autant
 qu'en icelui est descrite l'origine \& procreation de toutes choses :
 \& nommément des Peres anciens, qui ont esté tant devant qu'apres le deluge,
 \& eu esgard à IESVS CHRIST descendu d'iceux selon la chair.
\end{comment}
\vspace{1cm}
\end{@twocolumnfalse}
]

\vspace{3cm}

\bchapter

\begin{comment}
 \chapnote{Ce premier cha\-pitre est fort difficile: \& pour cette cause,
 il estoit defendu entre les Hebrieux de le lire \& interpreter
 devant l'aage de trente ans.}Creation du ciel \& de la terre,
 II, 10. \& de tout ce qui y est comprins.
 3.14. De la lumiere aussi, 26 \& de l'homme,
 18 Auquel tout est assuietti.
 2.2. 18 Dieu benit toutes ses \oe{}uvres,
 31 qu'il a accomplies en six iours.
\end{comment}

% TBD: span lettrine on 10 lines
\bversenonum \lettrine[lines=10,image=true]{D}{}%
 \versenote{\bibleverse{Job}(38:4), \bibleverse{Ps}(33:6), \& 89.13}Ieu
 \sidenote{Fit de rien, \& sans aucune matiere.}crea 
 \sidenote{Tout premierement, \& av\~at qu'il y eut aucune
 creature}au commencement
 \sidenote{Le ciel \& la terre,
 les eaux, les abysmes, se prennent ici pour vne mesme chose:
 asç. pour une matiere c\~ofuse \& sans forme, \~q Dieu forma
 \& agença apres par sa Parole.}le ciel \& la terre.
\bversenopar Or la terre estoit sans forme, \& vuide, \& les tenebres estoyent sur les
abysmes : \& l'Esprit de Dieu
 \sidenote{Ou, se mouvoit. C'est, soustenoit et conservoit
 en son estre cette matiere confuse.
 Car il est impossible, \~q aucune chose apres avoir esté
 faictes, puisse subsister un seul moment, si Dieu ne la soustient
 \& c\~oserve par sa vertu, \bibleverse{Ps}(130:).}estoit
 espandu par dessus les eaux.

\bverse Adonc Dieu dît,
 \versenote{\bibleverse{He}(11:3)}Qu'il y ait lumiere.
 \sidenote{Cette lumiere n'estoit point encore au soleil, car
 il n'avoit pas esté creé, mais estoit en la main de Dieu,
 ay\~at son ordre successif avec les tenebres,
 pour faire le iour \& la nuict \& ce iusques au
 quatrieme iour, que Dieu fit le soleil pour estre
 ministre \& dispensateur de cette lumiere, avec
 la lune \& estoilles.}Et la lumiere fut.

\bverse Et Dieu vid \~q la lumiere estoit bonne :
 \& separa la lumiere des tenebres.

\bverse Et Dieu appela la lumiere iour, \& les
 tenebres nuict. Lors fut faict le soir \& le matin 
 du premier iour.

\bverse ¶ Puis Dieu dît,
 \versenote{\bibleverse{Ps}(33:6) \& 136.5}Qu'il y ait une estendue entre les eaux,
 \& qu'elle separe les eaux d'avec les eaux.

\bverse Dieu donc fit l'estendue, \& divisa les eaux, qui estoyent sous
 l'estendue, d'avec celles, qui estoyent sur l'estendue. Et fut ainsi faict.

\bverse Et Dieu appela l'estendue, Ciel.
 Lors fut faict le soir \& le matin du second iour.

\bverse ¶ Puis Dieu dît, 
 \versenote{\bibleverse{Ps}(33:7)}Que les eaux, qui sont sous le ciel,
 soyent assemblees en un lieu, \& que le sec apparoisse. Et fut ainsi faict.

\bverse Et Dieu appela le sec, Terre, \& l'assemblee des eaux, mers.
 Et Dieu vid que celà estoit bon.

\bverse Et Dieu dît, Que la terre produise verdure, herbe produisant semence,
 \& arbre fruictier, faisant fruict selon son espece, lequel ait sa sem\~ece
 en soy-mesme sur la terre. Et fut ainsi faict.

\bverse La terre d\~oc produisit verdure, herbe produisant sem\~ece
 selon son espece, \& arbre sans fruict, lequel avoit sa semence en soymesme
 selon son espece. Et Dieu vid que celà estoit bon.

\bverse Lors fut faict le soir \& le matin du troisieme iour.

\bverse ¶ Apres Dieu dît, Qu'il y ait luminaires en l'estendue du ciel, pour
 separer la nuict du iour: \& soy\~et en signes, en

\end{document}

